% Options for packages loaded elsewhere
\PassOptionsToPackage{unicode}{hyperref}
\PassOptionsToPackage{hyphens}{url}
%
\documentclass[
]{article}
\usepackage{amsmath,amssymb}
\usepackage{lmodern}
\usepackage{ifxetex,ifluatex}
\ifnum 0\ifxetex 1\fi\ifluatex 1\fi=0 % if pdftex
  \usepackage[T1]{fontenc}
  \usepackage[utf8]{inputenc}
  \usepackage{textcomp} % provide euro and other symbols
\else % if luatex or xetex
  \usepackage{unicode-math}
  \defaultfontfeatures{Scale=MatchLowercase}
  \defaultfontfeatures[\rmfamily]{Ligatures=TeX,Scale=1}
\fi
% Use upquote if available, for straight quotes in verbatim environments
\IfFileExists{upquote.sty}{\usepackage{upquote}}{}
\IfFileExists{microtype.sty}{% use microtype if available
  \usepackage[]{microtype}
  \UseMicrotypeSet[protrusion]{basicmath} % disable protrusion for tt fonts
}{}
\makeatletter
\@ifundefined{KOMAClassName}{% if non-KOMA class
  \IfFileExists{parskip.sty}{%
    \usepackage{parskip}
  }{% else
    \setlength{\parindent}{0pt}
    \setlength{\parskip}{6pt plus 2pt minus 1pt}}
}{% if KOMA class
  \KOMAoptions{parskip=half}}
\makeatother
\usepackage{xcolor}
\IfFileExists{xurl.sty}{\usepackage{xurl}}{} % add URL line breaks if available
\IfFileExists{bookmark.sty}{\usepackage{bookmark}}{\usepackage{hyperref}}
\hypersetup{
  pdftitle={Statistical Computing HW3},
  pdfauthor={周聖諺},
  hidelinks,
  pdfcreator={LaTeX via pandoc}}
\urlstyle{same} % disable monospaced font for URLs
\usepackage[margin=1in]{geometry}
\usepackage{color}
\usepackage{fancyvrb}
\newcommand{\VerbBar}{|}
\newcommand{\VERB}{\Verb[commandchars=\\\{\}]}
\DefineVerbatimEnvironment{Highlighting}{Verbatim}{commandchars=\\\{\}}
% Add ',fontsize=\small' for more characters per line
\usepackage{framed}
\definecolor{shadecolor}{RGB}{248,248,248}
\newenvironment{Shaded}{\begin{snugshade}}{\end{snugshade}}
\newcommand{\AlertTok}[1]{\textcolor[rgb]{0.94,0.16,0.16}{#1}}
\newcommand{\AnnotationTok}[1]{\textcolor[rgb]{0.56,0.35,0.01}{\textbf{\textit{#1}}}}
\newcommand{\AttributeTok}[1]{\textcolor[rgb]{0.77,0.63,0.00}{#1}}
\newcommand{\BaseNTok}[1]{\textcolor[rgb]{0.00,0.00,0.81}{#1}}
\newcommand{\BuiltInTok}[1]{#1}
\newcommand{\CharTok}[1]{\textcolor[rgb]{0.31,0.60,0.02}{#1}}
\newcommand{\CommentTok}[1]{\textcolor[rgb]{0.56,0.35,0.01}{\textit{#1}}}
\newcommand{\CommentVarTok}[1]{\textcolor[rgb]{0.56,0.35,0.01}{\textbf{\textit{#1}}}}
\newcommand{\ConstantTok}[1]{\textcolor[rgb]{0.00,0.00,0.00}{#1}}
\newcommand{\ControlFlowTok}[1]{\textcolor[rgb]{0.13,0.29,0.53}{\textbf{#1}}}
\newcommand{\DataTypeTok}[1]{\textcolor[rgb]{0.13,0.29,0.53}{#1}}
\newcommand{\DecValTok}[1]{\textcolor[rgb]{0.00,0.00,0.81}{#1}}
\newcommand{\DocumentationTok}[1]{\textcolor[rgb]{0.56,0.35,0.01}{\textbf{\textit{#1}}}}
\newcommand{\ErrorTok}[1]{\textcolor[rgb]{0.64,0.00,0.00}{\textbf{#1}}}
\newcommand{\ExtensionTok}[1]{#1}
\newcommand{\FloatTok}[1]{\textcolor[rgb]{0.00,0.00,0.81}{#1}}
\newcommand{\FunctionTok}[1]{\textcolor[rgb]{0.00,0.00,0.00}{#1}}
\newcommand{\ImportTok}[1]{#1}
\newcommand{\InformationTok}[1]{\textcolor[rgb]{0.56,0.35,0.01}{\textbf{\textit{#1}}}}
\newcommand{\KeywordTok}[1]{\textcolor[rgb]{0.13,0.29,0.53}{\textbf{#1}}}
\newcommand{\NormalTok}[1]{#1}
\newcommand{\OperatorTok}[1]{\textcolor[rgb]{0.81,0.36,0.00}{\textbf{#1}}}
\newcommand{\OtherTok}[1]{\textcolor[rgb]{0.56,0.35,0.01}{#1}}
\newcommand{\PreprocessorTok}[1]{\textcolor[rgb]{0.56,0.35,0.01}{\textit{#1}}}
\newcommand{\RegionMarkerTok}[1]{#1}
\newcommand{\SpecialCharTok}[1]{\textcolor[rgb]{0.00,0.00,0.00}{#1}}
\newcommand{\SpecialStringTok}[1]{\textcolor[rgb]{0.31,0.60,0.02}{#1}}
\newcommand{\StringTok}[1]{\textcolor[rgb]{0.31,0.60,0.02}{#1}}
\newcommand{\VariableTok}[1]{\textcolor[rgb]{0.00,0.00,0.00}{#1}}
\newcommand{\VerbatimStringTok}[1]{\textcolor[rgb]{0.31,0.60,0.02}{#1}}
\newcommand{\WarningTok}[1]{\textcolor[rgb]{0.56,0.35,0.01}{\textbf{\textit{#1}}}}
\usepackage{graphicx}
\makeatletter
\def\maxwidth{\ifdim\Gin@nat@width>\linewidth\linewidth\else\Gin@nat@width\fi}
\def\maxheight{\ifdim\Gin@nat@height>\textheight\textheight\else\Gin@nat@height\fi}
\makeatother
% Scale images if necessary, so that they will not overflow the page
% margins by default, and it is still possible to overwrite the defaults
% using explicit options in \includegraphics[width, height, ...]{}
\setkeys{Gin}{width=\maxwidth,height=\maxheight,keepaspectratio}
% Set default figure placement to htbp
\makeatletter
\def\fps@figure{htbp}
\makeatother
\setlength{\emergencystretch}{3em} % prevent overfull lines
\providecommand{\tightlist}{%
  \setlength{\itemsep}{0pt}\setlength{\parskip}{0pt}}
\setcounter{secnumdepth}{-\maxdimen} % remove section numbering
\usepackage{xeCJK}
\usepackage{fontspec}
\setCJKmainfont{微軟正黑體}
\XeTeXlinebreaklocale "zh"
\XeTeXlinebreakskip = 0pt plus 1pt
\ifluatex
  \usepackage{selnolig}  % disable illegal ligatures
\fi

\title{Statistical Computing HW3}
\author{周聖諺}
\date{4/23/2021}

\begin{document}
\maketitle

\hypertarget{problem-1}{%
\subsection{Problem 1:}\label{problem-1}}

Two random variables are defined \(X_1, X_2\) and combined as a set
\(X = \{ X_1, X_2 \}\)

\[
X_1 = \sigma_{X_1} Z_1 + \mu_{X_1}
\]

\[
X_2 = \sigma_{X_2} \Big(\rho Z_1 + \sqrt{1 - \rho^2} Z_2 \Big)  + \mu_{X_2}
\]

where \(Z_1, Z_2 \sim \mathcal{N}(0, 1)\) and \(Z_1, Z_2\) are
independent

The Expectation

\[
\mathbb{E}[X_1] = \mathbb{E}[\sigma_{X_1} Z_1 + \mu_{X_1}] = \sigma_{X_1} \mathbb{E}[Z_1] + \mu_{X_1} = \mu_{X_1}
\]

\[
\mathbb{E}[X_2] = \mathbb{E}[\sigma_{X_2} (\rho Z_1 + \sqrt{1 - \rho^2} Z_2  + \mu_{X_2}]
\]

\[
= \sigma_{X_2} \mathbb{E}[\rho Z_1 + \sqrt{1 - \rho^2} Z_2]  + \mu_{X_2}
\]

\[
= \sigma_{X_2} (\rho \mathbb{E}[Z_1] + \sqrt{1 - \rho^2} \mathbb{E}[Z_2])  + \mu_{X_2} = \mu_{X_2}
\]

\[
\mathbb{E}[X] = \{ \mu_{X_1}, \mu_{X_2} \}
\]

The Covariance

\[
\sigma_{X_1, X_2} = \sigma_{X_2, X_1} = \mathbb{E}[(X_1 - \mu_{X_1})(X_2 - \mu_{X_2})]
\] \[
= \mathbb{E}[(\sigma_{X_1} Z_1 + \mu_{X_1} - \mu_{X_1})(\sigma_{X_2} (\rho Z_1 + \sqrt{1 - \rho^2} Z_2  + \mu_{X_2} - \mu_{X_2})]
\] \[
= \mathbb{E}[(\sigma_{X_1} Z_1)(\sigma_{X_2} (\rho Z_1 + \sqrt{1 - \rho^2} Z_2)]
\] \[
= \mathbb{E}[\sigma_{X_1} \sigma_{X_2}( \rho Z_1^2 + \sqrt{1 - \rho^2} Z_1 Z_2)]
\]

\[
= \sigma_{X_1} \sigma_{X_2} \Big( \rho \mathbb{E}[Z_1^2] + \sqrt{1 - \rho^2} \mathbb{E}[Z_1 Z_2] \Big)
\] With the definition of variance, we can derive
\(\mathbb{E}[Z_1^2] = Var[Z_1] + E[Z_1]^2 = 1\). Since \(Z_1, Z_2\) are
independent,
\(\mathbb{E}[Z_1 Z_2] = \mathbb{E}[Z_1] \mathbb{E}[Z_2] = 0\)

\[
= \sigma_{X_1} \sigma_{X_2} \rho
\] Then, conduct a simulation as pseudo code

\begin{center}\rule{0.5\linewidth}{0.5pt}\end{center}

\(Z_1\) = \mathcal{N}(0, 1) \(Z_2\) = \mathcal{N}(0, 1)

\(X_1 = \sigma_{X_1} Z_1 + \mu_{X_1}\)
\(X_2 = \sigma_{X_2} (\rho Z_1 + \sqrt{1 - \rho^2} Z_2 + \mu_{X_2}\)

return \(X_1, X_2\)

\begin{center}\rule{0.5\linewidth}{0.5pt}\end{center}

\begin{verbatim}
## Mean of X_1:  0.9989595  
## NULL
\end{verbatim}

\begin{verbatim}
## Mean of X_2:  1.987382  
## NULL
\end{verbatim}

\begin{verbatim}
## Covariance of (X_1, X_2):  0.818182  
## NULL
\end{verbatim}

\hypertarget{problem-2}{%
\subsection{Problem 2:}\label{problem-2}}

The conditional probability of bivariate normal distribution

\[
\begin{eqnarray*} \begin{pmatrix} X\\ Y \end{pmatrix} & \sim & N\left[\left(\begin{array}{c} \mu_X\\ \mu_Y \end{array}\right),\left(\begin{array}{ccc} \sigma_X^2 & \rho \sigma_X \sigma_Y\\ \rho \sigma_X \sigma_Y & \sigma_Y^2 \end{array}\right)\right]\\ \end{eqnarray*}
\]

\[
Y|X = N\left(\mu_Y+\rho \frac{\sigma_Y}{\sigma_X}(X-\mu_X),\ \sigma^2_Y(1-\rho^2)\right)
\]

\[
= \mu_Y+\rho \frac{\sigma_Y}{\sigma_X}(X-\mu_X) + \sigma_Y \sqrt{(1-\rho^2)} \mathcal{N}(0, 1)
\]

\textbf{Gibbs Sampling}

\begin{center}\rule{0.5\linewidth}{0.5pt}\end{center}

For each k-th smapling

\begin{itemize}
\item
  \(X_{1}^{k} = \mu_{X_2}+\rho \frac{\sigma_{X_2}}{\sigma_{X_1}}(X_2^{k-1}-\mu_{X_1}) + \sigma_{X_2} \sqrt{(1-\rho^2)} \mathcal{N}(0, 1)\)
\item
  \(X_2^{k} = \mu_{X_1}+\rho \frac{\sigma_{X_1}}{\sigma_{X_2}}(X_1^{k}-\mu_{X_2}) + \sigma_{X_1} \sqrt{(1-\rho^2)} \mathcal{N}(0, 1)\)
\end{itemize}

Return \(\{ \{ X_1^1, ..., X_1^N \}, \{ X_2^1, ..., X_2^N \} \}\)

\begin{center}\rule{0.5\linewidth}{0.5pt}\end{center}

\begin{verbatim}
## Mean of Y1:  1.581207NULL
\end{verbatim}

\begin{verbatim}
## Mean of Y2:  1.540082NULL
\end{verbatim}

\begin{verbatim}
## [1] "Covariance Matrix"
\end{verbatim}

\begin{verbatim}
##           [,1]      [,2]
## [1,] 2.8111990 0.7648705
## [2,] 0.7648705 3.0533501
\end{verbatim}

\begin{verbatim}
## Number of data point:  100NULL
\end{verbatim}

\includegraphics{hw3_files/figure-latex/q2-1.pdf}
\includegraphics{hw3_files/figure-latex/q2-2.pdf}

\hypertarget{problem-3}{%
\subsection{Problem 3:}\label{problem-3}}

\hypertarget{generate-data}{%
\subsubsection{Generate Data}\label{generate-data}}

\includegraphics{hw3_files/figure-latex/q3_gen_data-1.pdf}

\hypertarget{k-means}{%
\subsubsection{(1) K-Means}\label{k-means}}

\begin{center}\rule{0.5\linewidth}{0.5pt}\end{center}

Repeat until converge

\begin{itemize}
\item
  For each data point, Compute the distance between the data point and
  the nearest cluster center
\item
  Assign the data point to the nearest cluster
\item
  Compute the distance between each data point and each cluster center
  and check whether it converge or not
\end{itemize}

\begin{center}\rule{0.5\linewidth}{0.5pt}\end{center}

\begin{verbatim}
## Warning in cbind(dt_1, dt_2): number of rows of result is not a multiple of
## vector length (arg 2)
\end{verbatim}

\includegraphics{hw3_files/figure-latex/q3_kmean-1.pdf}

\hypertarget{em-gmm}{%
\subsubsection{(2) EM-GMM}\label{em-gmm}}

\begin{center}\rule{0.5\linewidth}{0.5pt}\end{center}

Repeat until converge

\begin{itemize}
\item
  E Step

  \begin{itemize}
  \tightlist
  \item
    Compute the likelihood \(\mathcal{L}_{ij}\) of i-th data point and
    j-th component
  \item
    Multiply the likelihood \(\mathcal{L}_{ij}\) with the weight
    \(w_{j}\) and divide the likelihood \(\mathcal{L}_{ij}\) by the sum
    of the i-th weighted likelihood
    \(\sum_{j=1}^K w_{j} \mathcal{L}_{ij}\). Thus, we can derive
    \(\gamma_{ij}\).
    \(\gamma_{ij} = \frac{w_{j} \mathcal{L}_{ij}}{\sum_{j=1}^K w_{j} \mathcal{L}_{ij}}\)
  \end{itemize}
\item ~
  \hypertarget{m-step}{%
  \subsection{M Step}\label{m-step}}
\end{itemize}

\begin{center}\rule{0.5\linewidth}{0.5pt}\end{center}

\begin{verbatim}
##        Component  1 Component  2
## ws         0.954995   0.04500504
## mus        1.027466   0.59667923
## sigmas     1.875652   0.05761412
\end{verbatim}

\includegraphics{hw3_files/figure-latex/q3_em-gmm-1.pdf}

\hypertarget{code}{%
\section{Code}\label{code}}

\hypertarget{problem-1-1}{%
\subsection{Problem 1}\label{problem-1-1}}

\begin{Shaded}
\begin{Highlighting}[]
\NormalTok{gen\_binorm }\OtherTok{\textless{}{-}} \ControlFlowTok{function}\NormalTok{(n, mu\_1, mu\_2, sigma\_1, sigma\_2, rho)\{}
\NormalTok{  z\_1 }\OtherTok{\textless{}{-}} \FunctionTok{rnorm}\NormalTok{(n, }\DecValTok{0}\NormalTok{, }\DecValTok{1}\NormalTok{)}
\NormalTok{  z\_2 }\OtherTok{\textless{}{-}} \FunctionTok{rnorm}\NormalTok{(n, }\DecValTok{0}\NormalTok{, }\DecValTok{1}\NormalTok{)}
\NormalTok{  z\_bind }\OtherTok{\textless{}{-}} \FunctionTok{rbind}\NormalTok{(z\_1, z\_2)}
  
\NormalTok{  x\_1 }\OtherTok{\textless{}{-}} \FunctionTok{sapply}\NormalTok{(z\_1, }\ControlFlowTok{function}\NormalTok{(z)\{}\FunctionTok{return}\NormalTok{(sigma\_1 }\SpecialCharTok{*}\NormalTok{ z }\SpecialCharTok{+}\NormalTok{ mu\_1)\})}
\NormalTok{  x\_2 }\OtherTok{\textless{}{-}} \FunctionTok{apply}\NormalTok{(z\_bind, }\DecValTok{2}\NormalTok{, }\ControlFlowTok{function}\NormalTok{(z)\{}\FunctionTok{return}\NormalTok{(sigma\_2 }\SpecialCharTok{*}\NormalTok{ (rho }\SpecialCharTok{*}\NormalTok{ z[}\DecValTok{1}\NormalTok{] }\SpecialCharTok{+} \FunctionTok{sqrt}\NormalTok{(}\DecValTok{1} \SpecialCharTok{{-}}\NormalTok{ rho }\SpecialCharTok{*}\NormalTok{ rho) }\SpecialCharTok{*}\NormalTok{ z[}\DecValTok{2}\NormalTok{]) }\SpecialCharTok{+}\NormalTok{ mu\_2)\})}
  
\NormalTok{  x\_bind }\OtherTok{\textless{}{-}} \FunctionTok{rbind}\NormalTok{(x\_1, x\_2)}
  \FunctionTok{return}\NormalTok{(x\_bind)}
\NormalTok{\}}

\NormalTok{x }\OtherTok{\textless{}{-}} \FunctionTok{gen\_binorm}\NormalTok{(}\DecValTok{5000}\NormalTok{, }\DecValTok{1}\NormalTok{, }\DecValTok{2}\NormalTok{, }\DecValTok{1}\NormalTok{, }\DecValTok{2}\NormalTok{, }\FloatTok{0.4}\NormalTok{)}

\FunctionTok{print}\NormalTok{(}\FunctionTok{cat}\NormalTok{(}\StringTok{"Mean of X\_1: "}\NormalTok{, }\FunctionTok{format}\NormalTok{(}\FunctionTok{mean}\NormalTok{(x[}\DecValTok{1}\NormalTok{, ])), }\StringTok{" }\SpecialCharTok{\textbackslash{}n}\StringTok{"}\NormalTok{))}
\FunctionTok{print}\NormalTok{(}\FunctionTok{cat}\NormalTok{(}\StringTok{"Mean of X\_2: "}\NormalTok{, }\FunctionTok{format}\NormalTok{(}\FunctionTok{mean}\NormalTok{(x[}\DecValTok{2}\NormalTok{, ])), }\StringTok{" }\SpecialCharTok{\textbackslash{}n}\StringTok{"}\NormalTok{))}
\FunctionTok{print}\NormalTok{(}\FunctionTok{cat}\NormalTok{(}\StringTok{"Covariance of (X\_1, X\_2): "}\NormalTok{, }\FunctionTok{format}\NormalTok{(}\FunctionTok{cov}\NormalTok{(x[}\DecValTok{1}\NormalTok{, ], x[}\DecValTok{2}\NormalTok{, ])), }\StringTok{" }\SpecialCharTok{\textbackslash{}n}\StringTok{"}\NormalTok{))}
\end{Highlighting}
\end{Shaded}

\hypertarget{problem-2-1}{%
\subsection{Problem 2}\label{problem-2-1}}

\begin{Shaded}
\begin{Highlighting}[]
\FunctionTok{library}\NormalTok{(Rlab)}
\end{Highlighting}
\end{Shaded}

\begin{Shaded}
\begin{Highlighting}[]
\NormalTok{binorm }\OtherTok{\textless{}{-}} \ControlFlowTok{function}\NormalTok{(n, mus, sigmas, rho, }\AttributeTok{warmup=}\DecValTok{10000}\NormalTok{)\{}
  \CommentTok{\# Dimension}
\NormalTok{  d }\OtherTok{\textless{}{-}} \FunctionTok{length}\NormalTok{(mus)}
\NormalTok{  total\_n }\OtherTok{\textless{}{-}}\NormalTok{ n}\SpecialCharTok{+}\NormalTok{warmup}
\NormalTok{  rvs }\OtherTok{\textless{}{-}} \FunctionTok{matrix}\NormalTok{(}\DecValTok{0}\NormalTok{, }\AttributeTok{nrow =}\NormalTok{ total\_n, }\AttributeTok{ncol =}\NormalTok{ d)}
  
  \ControlFlowTok{for}\NormalTok{(i }\ControlFlowTok{in} \DecValTok{2}\SpecialCharTok{:}\NormalTok{total\_n)\{}
    \CommentTok{\# Generate X\_1}
\NormalTok{    rvs[i, }\DecValTok{2}\NormalTok{] }\OtherTok{\textless{}{-}}\NormalTok{ mus[}\DecValTok{2}\NormalTok{] }\SpecialCharTok{+}\NormalTok{ (rho }\SpecialCharTok{*}\NormalTok{ sigmas[}\DecValTok{2}\NormalTok{] }\SpecialCharTok{/}\NormalTok{ sigmas[}\DecValTok{1}\NormalTok{] }\SpecialCharTok{*}\NormalTok{ (rvs[i}\DecValTok{{-}1}\NormalTok{, }\DecValTok{1}\NormalTok{] }\SpecialCharTok{{-}}\NormalTok{ mus[}\DecValTok{1}\NormalTok{])) }\SpecialCharTok{+}\NormalTok{ sigmas[}\DecValTok{2}\NormalTok{] }\SpecialCharTok{*} \FunctionTok{sqrt}\NormalTok{(}\DecValTok{1}\SpecialCharTok{{-}}\NormalTok{rho}\SpecialCharTok{\^{}}\DecValTok{2}\NormalTok{) }\SpecialCharTok{*} \FunctionTok{rnorm}\NormalTok{(}\DecValTok{1}\NormalTok{, }\DecValTok{0}\NormalTok{, }\DecValTok{1}\NormalTok{)}
    
    \CommentTok{\# Generate X\_2}
\NormalTok{    rvs[i, }\DecValTok{1}\NormalTok{] }\OtherTok{\textless{}{-}}\NormalTok{ mus[}\DecValTok{1}\NormalTok{] }\SpecialCharTok{+}\NormalTok{ (rho }\SpecialCharTok{*}\NormalTok{ sigmas[}\DecValTok{1}\NormalTok{] }\SpecialCharTok{/}\NormalTok{ sigmas[}\DecValTok{2}\NormalTok{] }\SpecialCharTok{*}\NormalTok{ (rvs[i, }\DecValTok{2}\NormalTok{] }\SpecialCharTok{{-}}\NormalTok{ mus[}\DecValTok{2}\NormalTok{])) }\SpecialCharTok{+}\NormalTok{ sigmas[}\DecValTok{1}\NormalTok{] }\SpecialCharTok{*} \FunctionTok{sqrt}\NormalTok{(}\DecValTok{1}\SpecialCharTok{{-}}\NormalTok{rho}\SpecialCharTok{\^{}}\DecValTok{2}\NormalTok{) }\SpecialCharTok{*} \FunctionTok{rnorm}\NormalTok{(}\DecValTok{1}\NormalTok{, }\DecValTok{0}\NormalTok{, }\DecValTok{1}\NormalTok{)}
\NormalTok{  \}}
  
  \FunctionTok{return}\NormalTok{(rvs[(warmup}\SpecialCharTok{+}\DecValTok{1}\NormalTok{)}\SpecialCharTok{:}\NormalTok{total\_n, ])}
\NormalTok{\}}

\NormalTok{mixture\_binorm }\OtherTok{\textless{}{-}} \ControlFlowTok{function}\NormalTok{(n, mus, sigmas, rhos, ws)\{}
\NormalTok{  modals }\OtherTok{\textless{}{-}} \FunctionTok{length}\NormalTok{(mus[, }\DecValTok{1}\NormalTok{])}
\NormalTok{  d }\OtherTok{\textless{}{-}} \FunctionTok{length}\NormalTok{(mus[}\DecValTok{1}\NormalTok{, ])}
\NormalTok{  total\_rvs }\OtherTok{\textless{}{-}} \FunctionTok{array}\NormalTok{(}\DecValTok{0}\NormalTok{, }\FunctionTok{c}\NormalTok{(n, d, modals))}
\NormalTok{  inds }\OtherTok{\textless{}{-}} \FunctionTok{rbern}\NormalTok{(n, ws[}\DecValTok{2}\NormalTok{])}
  
  \CommentTok{\#print(modals)}
  \CommentTok{\#print(d)}
  \CommentTok{\#print(inds)}
  \CommentTok{\#print(total\_rvs)}
  
  \ControlFlowTok{for}\NormalTok{(i }\ControlFlowTok{in} \DecValTok{1}\SpecialCharTok{:}\NormalTok{modals)\{}
\NormalTok{    total\_rvs[,,i] }\OtherTok{\textless{}{-}} \FunctionTok{binorm}\NormalTok{(n, mus[i, ], sigmas[i, ], rhos[i])}
\NormalTok{  \}}
\NormalTok{  i }\OtherTok{\textless{}{-}} \DecValTok{1}
  \CommentTok{\#print(total\_rvs)}
  
\NormalTok{  rvs }\OtherTok{\textless{}{-}} \FunctionTok{matrix}\NormalTok{(}\DecValTok{0}\NormalTok{, }\AttributeTok{ncol=}\NormalTok{d, }\AttributeTok{nrow=}\NormalTok{n)}
  \CommentTok{\#rvs[, 1] = rvs[, 2] \textless{}{-} c(1:n)}
  \ControlFlowTok{for}\NormalTok{(i }\ControlFlowTok{in} \DecValTok{1}\SpecialCharTok{:}\NormalTok{n)\{}
\NormalTok{    rvs[i, ] }\OtherTok{\textless{}{-}} \FunctionTok{c}\NormalTok{(total\_rvs[i,,inds[i]}\SpecialCharTok{+}\DecValTok{1}\NormalTok{])}
\NormalTok{  \}}
  
  \FunctionTok{return}\NormalTok{(rvs)}
\NormalTok{\}}

\CommentTok{\#res \textless{}{-} binorm(10000, c(1, 2), c(1, 2), 0.4)}
\NormalTok{res }\OtherTok{\textless{}{-}} \FunctionTok{mixture\_binorm}\NormalTok{(}\DecValTok{100}\NormalTok{, }\FunctionTok{matrix}\NormalTok{(}\FunctionTok{c}\NormalTok{(}\DecValTok{1}\NormalTok{, }\DecValTok{2}\NormalTok{, }\DecValTok{2}\NormalTok{, }\DecValTok{1}\NormalTok{), }\AttributeTok{ncol=}\DecValTok{2}\NormalTok{), }\FunctionTok{matrix}\NormalTok{(}\FunctionTok{c}\NormalTok{(}\DecValTok{1}\NormalTok{, }\DecValTok{2}\NormalTok{, }\DecValTok{2}\NormalTok{, }\DecValTok{1}\NormalTok{), }\AttributeTok{ncol=}\DecValTok{2}\NormalTok{), }\FunctionTok{c}\NormalTok{(}\FloatTok{0.4}\NormalTok{, }\FloatTok{0.6}\NormalTok{), }\FunctionTok{c}\NormalTok{(}\FloatTok{0.4}\NormalTok{, }\FloatTok{0.6}\NormalTok{))}

\FunctionTok{print}\NormalTok{(}\FunctionTok{cat}\NormalTok{(}\StringTok{"Mean of Y1: "}\NormalTok{, }\FunctionTok{mean}\NormalTok{(res[, }\DecValTok{1}\NormalTok{])))}
\FunctionTok{print}\NormalTok{(}\FunctionTok{cat}\NormalTok{(}\StringTok{"Mean of Y2: "}\NormalTok{, }\FunctionTok{mean}\NormalTok{(res[, }\DecValTok{2}\NormalTok{])))}
\FunctionTok{print}\NormalTok{(}\StringTok{"Covariance Matrix"}\NormalTok{)}
\FunctionTok{print}\NormalTok{(}\FunctionTok{cov}\NormalTok{(res))}
\FunctionTok{print}\NormalTok{(}\FunctionTok{cat}\NormalTok{(}\StringTok{"Number of data point: "}\NormalTok{, }\FunctionTok{length}\NormalTok{(res[, }\DecValTok{1}\NormalTok{])))}

\CommentTok{\# Scatter}
\FunctionTok{plot}\NormalTok{(res[,}\DecValTok{1}\NormalTok{], res[,}\DecValTok{2}\NormalTok{], }\AttributeTok{main=}\StringTok{"Scatter Plot"}\NormalTok{, }\AttributeTok{xlab=}\StringTok{"Y1"}\NormalTok{, }\AttributeTok{ylab=}\StringTok{"Y2"}\NormalTok{)}

\CommentTok{\# Dendrogram}
\NormalTok{dg }\OtherTok{\textless{}{-}} \FunctionTok{hclust}\NormalTok{(}\FunctionTok{dist}\NormalTok{(}\FunctionTok{scale}\NormalTok{(res), }\AttributeTok{method =} \StringTok{"euclidean"}\NormalTok{), }\AttributeTok{method =} \StringTok{"ward.D2"}\NormalTok{)}
\FunctionTok{plot}\NormalTok{(dg, }\AttributeTok{hang=}\FloatTok{0.1}\NormalTok{, }\AttributeTok{main=}\StringTok{"Cluster Dendrogram"}\NormalTok{, }\AttributeTok{sub=}\ConstantTok{NULL}\NormalTok{, }\AttributeTok{xlab=}\StringTok{"Points"}\NormalTok{, }\AttributeTok{ylab=}\StringTok{"Height"}\NormalTok{)}
\end{Highlighting}
\end{Shaded}

\hypertarget{problem-3-1}{%
\subsection{Problem 3}\label{problem-3-1}}

\hypertarget{a}{%
\subsubsection{(a)}\label{a}}

\begin{Shaded}
\begin{Highlighting}[]
\NormalTok{dis }\OtherTok{\textless{}{-}} \ControlFlowTok{function}\NormalTok{(a, b)\{}
  \FunctionTok{return}\NormalTok{(}\FunctionTok{sum}\NormalTok{((a }\SpecialCharTok{{-}}\NormalTok{ b)}\SpecialCharTok{\^{}}\DecValTok{2}\NormalTok{))}
\NormalTok{\}}

\NormalTok{kmean }\OtherTok{\textless{}{-}} \ControlFlowTok{function}\NormalTok{(X, K, }\AttributeTok{threshold=}\FloatTok{1e{-}6}\NormalTok{, }\AttributeTok{max\_iter=}\DecValTok{10}\NormalTok{)\{}
\NormalTok{  n }\OtherTok{\textless{}{-}} \FunctionTok{length}\NormalTok{(X)}
\NormalTok{  d }\OtherTok{\textless{}{-}} \DecValTok{1}
\NormalTok{  cent\_idxs }\OtherTok{\textless{}{-}} \FunctionTok{sample}\NormalTok{(}\DecValTok{1}\SpecialCharTok{:}\NormalTok{n, K)}
\NormalTok{  cent\_cord }\OtherTok{\textless{}{-}} \FunctionTok{matrix}\NormalTok{(}\DecValTok{0}\NormalTok{, }\AttributeTok{nrow=}\NormalTok{K, }\AttributeTok{ncol=}\NormalTok{d)}
\NormalTok{  dis\_table }\OtherTok{\textless{}{-}} \FunctionTok{matrix}\NormalTok{(}\DecValTok{0}\NormalTok{, }\AttributeTok{nrow=}\NormalTok{n, }\AttributeTok{ncol=}\NormalTok{K)}
\NormalTok{  clustered }\OtherTok{\textless{}{-}} \FunctionTok{array}\NormalTok{(}\DecValTok{0}\NormalTok{, }\FunctionTok{c}\NormalTok{(n))}
\NormalTok{  tot\_dis }\OtherTok{\textless{}{-}} \ConstantTok{Inf}
  
  \ControlFlowTok{for}\NormalTok{(k }\ControlFlowTok{in} \DecValTok{1}\SpecialCharTok{:}\NormalTok{K)\{}
\NormalTok{    cent\_cord[k] }\OtherTok{\textless{}{-}}\NormalTok{ X[cent\_idxs[k]]}
\NormalTok{  \}}
  
  \ControlFlowTok{for}\NormalTok{(iter }\ControlFlowTok{in} \DecValTok{1}\SpecialCharTok{:}\NormalTok{max\_iter)\{}
    \CommentTok{\# Calculate distance between every pairs of cluster center and data point}
    \ControlFlowTok{for}\NormalTok{(k }\ControlFlowTok{in} \DecValTok{1}\SpecialCharTok{:}\NormalTok{K)\{}
\NormalTok{      dis\_table[, k] }\OtherTok{\textless{}{-}} \FunctionTok{sapply}\NormalTok{(X, dis, cent\_cord[k])}
\NormalTok{    \}}
    \CommentTok{\# Assign the nearest cluster}
\NormalTok{    clustered }\OtherTok{\textless{}{-}} \FunctionTok{apply}\NormalTok{(dis\_table, }\DecValTok{1}\NormalTok{, which.min)}
    \CommentTok{\# Update the center}
    \ControlFlowTok{for}\NormalTok{(k }\ControlFlowTok{in} \DecValTok{1}\SpecialCharTok{:}\NormalTok{K)\{}
\NormalTok{      idxs }\OtherTok{\textless{}{-}} \FunctionTok{which}\NormalTok{(clustered }\SpecialCharTok{==}\NormalTok{ k)}
\NormalTok{      cent\_cord[k] }\OtherTok{\textless{}{-}} \FunctionTok{mean}\NormalTok{(X[idxs])}
      
      \CommentTok{\#print(idxs)}
      \CommentTok{\#print(X[idxs])}
\NormalTok{    \}}
    \CommentTok{\# Calculate total distance}
\NormalTok{    temp\_tot\_dis }\OtherTok{\textless{}{-}} \FunctionTok{sum}\NormalTok{(dis\_table)}
    \ControlFlowTok{if}\NormalTok{(}\FunctionTok{abs}\NormalTok{(temp\_tot\_dis }\SpecialCharTok{{-}}\NormalTok{ tot\_dis) }\SpecialCharTok{\textless{}}\NormalTok{ threshold)\{}\ControlFlowTok{break}\NormalTok{\}}
\NormalTok{    tot\_dis }\OtherTok{\textless{}{-}}\NormalTok{ temp\_tot\_dis}
    
    \CommentTok{\#print(dis\_table)}
    \CommentTok{\#print(clustered)}
    \CommentTok{\#print(tot\_dis)}
\NormalTok{  \}}
  
  \CommentTok{\#print(clustered)}
  \CommentTok{\#print(tot\_dis)}
  
\NormalTok{  res }\OtherTok{\textless{}{-}} \FunctionTok{structure}\NormalTok{(}\FunctionTok{list}\NormalTok{(}\AttributeTok{data=}\NormalTok{X, }\AttributeTok{cluster=}\NormalTok{clustered, }\AttributeTok{total\_distance=}\NormalTok{tot\_dis, }\AttributeTok{cluster\_centers=}\NormalTok{cent\_cord), }\AttributeTok{class=} \StringTok{"KMEAN\_res"}\NormalTok{)}
    
  \FunctionTok{return}\NormalTok{(res)}
\NormalTok{\}}

\NormalTok{kmean\_model }\OtherTok{\textless{}{-}} \FunctionTok{kmean}\NormalTok{(data, }\DecValTok{2}\NormalTok{)}

\NormalTok{dt\_1 }\OtherTok{\textless{}{-}}\NormalTok{ data[}\FunctionTok{which}\NormalTok{(kmean\_model}\SpecialCharTok{$}\NormalTok{cluster }\SpecialCharTok{==} \DecValTok{1}\NormalTok{)]}
\NormalTok{dt\_2 }\OtherTok{\textless{}{-}}\NormalTok{ data[}\FunctionTok{which}\NormalTok{(kmean\_model}\SpecialCharTok{$}\NormalTok{cluster }\SpecialCharTok{==} \DecValTok{2}\NormalTok{)]}
\NormalTok{dt }\OtherTok{\textless{}{-}} \FunctionTok{cbind}\NormalTok{(dt\_1, dt\_2)}
\FunctionTok{colnames}\NormalTok{(dt) }\OtherTok{\textless{}{-}} \FunctionTok{c}\NormalTok{(}\StringTok{"Cluster 1"}\NormalTok{, }\StringTok{"Cluster 2"}\NormalTok{)}

\FunctionTok{boxplot}\NormalTok{(dt, }\AttributeTok{n=}\DecValTok{2}\NormalTok{, }\AttributeTok{xlab=}\StringTok{"Clusters"}\NormalTok{, }\AttributeTok{ylab=}\StringTok{"Values"}\NormalTok{, }\AttributeTok{main=}\StringTok{"Box Plot of K{-}Means Clustering"}\NormalTok{)}
\end{Highlighting}
\end{Shaded}

\hypertarget{b}{%
\subsubsection{(b)}\label{b}}

\begin{Shaded}
\begin{Highlighting}[]
\NormalTok{e\_step }\OtherTok{\textless{}{-}} \ControlFlowTok{function}\NormalTok{(ys, ws, mus, sigmas)\{}
\NormalTok{  k }\OtherTok{\textless{}{-}} \FunctionTok{length}\NormalTok{(ws)}
\NormalTok{  n }\OtherTok{\textless{}{-}} \FunctionTok{length}\NormalTok{(ys)}
\NormalTok{  likelihoods }\OtherTok{\textless{}{-}} \FunctionTok{matrix}\NormalTok{(}\FunctionTok{rep}\NormalTok{(}\DecValTok{0}\NormalTok{, k}\SpecialCharTok{*}\NormalTok{n), }\AttributeTok{nrow =}\NormalTok{ n)}
\NormalTok{  weighted\_likelihoods }\OtherTok{\textless{}{-}} \FunctionTok{matrix}\NormalTok{(}\FunctionTok{rep}\NormalTok{(}\DecValTok{0}\NormalTok{, k}\SpecialCharTok{*}\NormalTok{n), }\AttributeTok{nrow =}\NormalTok{ n)}
  
  \CommentTok{\# Evaluate the hidden variables}
  \ControlFlowTok{for}\NormalTok{(j }\ControlFlowTok{in} \DecValTok{1}\SpecialCharTok{:}\NormalTok{k)\{}
\NormalTok{    likelihoods[, j] }\OtherTok{\textless{}{-}} \FunctionTok{sapply}\NormalTok{(ys, dnorm, mus[j], sigmas[j])}
\NormalTok{    weighted\_likelihoods[, j] }\OtherTok{\textless{}{-}}\NormalTok{ ws[j] }\SpecialCharTok{*}\NormalTok{ likelihoods[, j]}
\NormalTok{  \}}
  \CommentTok{\# gamma\_i}
\NormalTok{  weighted\_likelihoods }\OtherTok{\textless{}{-}}\NormalTok{ weighted\_likelihoods }\SpecialCharTok{/} \FunctionTok{rowSums}\NormalTok{(weighted\_likelihoods)}
  
  \FunctionTok{return}\NormalTok{(weighted\_likelihoods)}
\NormalTok{\}}

\NormalTok{m\_step }\OtherTok{\textless{}{-}} \ControlFlowTok{function}\NormalTok{(ys, ws, mus, sigmas, gammas)\{}
\NormalTok{  k }\OtherTok{\textless{}{-}} \FunctionTok{length}\NormalTok{(ws)}
\NormalTok{  n }\OtherTok{\textless{}{-}} \FunctionTok{length}\NormalTok{(ys)}
  
  \CommentTok{\#Maximize the estimate}
  \ControlFlowTok{for}\NormalTok{(j }\ControlFlowTok{in} \DecValTok{1}\SpecialCharTok{:}\NormalTok{k)\{}
\NormalTok{    sum\_gammas }\OtherTok{\textless{}{-}} \FunctionTok{sum}\NormalTok{(gammas[, j])}
\NormalTok{    mus[j] }\OtherTok{\textless{}{-}} \FunctionTok{sum}\NormalTok{(ys }\SpecialCharTok{*}\NormalTok{ gammas[, j]) }\SpecialCharTok{/}\NormalTok{ sum\_gammas}
\NormalTok{    sigmas[j] }\OtherTok{\textless{}{-}} \FunctionTok{sqrt}\NormalTok{(}\FunctionTok{sum}\NormalTok{(gammas[, j] }\SpecialCharTok{*}\NormalTok{ (ys }\SpecialCharTok{{-}}\NormalTok{ mus[j])}\SpecialCharTok{\^{}}\DecValTok{2}\NormalTok{) }\SpecialCharTok{/}\NormalTok{ sum\_gammas)}
\NormalTok{    ws[j] }\OtherTok{\textless{}{-}} \FunctionTok{mean}\NormalTok{(gammas[, j])}
\NormalTok{  \}}
  
  \FunctionTok{return}\NormalTok{(}\FunctionTok{rbind}\NormalTok{(ws, mus, sigmas))}
\NormalTok{\}}

\NormalTok{em }\OtherTok{\textless{}{-}} \ControlFlowTok{function}\NormalTok{(ys, k, }\AttributeTok{threshold=}\FloatTok{1e{-}9}\NormalTok{, }\AttributeTok{max\_iter=}\DecValTok{201}\NormalTok{)\{}
\NormalTok{  mus }\OtherTok{\textless{}{-}} \FunctionTok{runif}\NormalTok{(k)}
\NormalTok{  sigmas }\OtherTok{\textless{}{-}} \FunctionTok{runif}\NormalTok{(k)}
\NormalTok{  ws }\OtherTok{\textless{}{-}} \FunctionTok{rep}\NormalTok{(}\DecValTok{1}\SpecialCharTok{/}\NormalTok{k, k)}
  
\NormalTok{  old\_params }\OtherTok{\textless{}{-}} \FunctionTok{rbind}\NormalTok{(ws, mus, sigmas)}

  \ControlFlowTok{for}\NormalTok{(i }\ControlFlowTok{in} \DecValTok{1}\SpecialCharTok{:}\NormalTok{max\_iter)\{}
\NormalTok{    gammas }\OtherTok{\textless{}{-}} \FunctionTok{e\_step}\NormalTok{(data, ws, mus, sigmas)}
\NormalTok{    params }\OtherTok{\textless{}{-}} \FunctionTok{m\_step}\NormalTok{(data, ws, mus, sigmas, gammas)}
    \CommentTok{\#print(gammas)}
    
    \CommentTok{\# Update parameters}
\NormalTok{    ws }\OtherTok{\textless{}{-}}\NormalTok{ params[}\DecValTok{1}\NormalTok{, ]}
\NormalTok{    mus }\OtherTok{\textless{}{-}}\NormalTok{ params[}\DecValTok{2}\NormalTok{, ]}
\NormalTok{    sigmas }\OtherTok{\textless{}{-}}\NormalTok{ params[}\DecValTok{3}\NormalTok{, ]}
    
    \CommentTok{\# Until converge}
    \ControlFlowTok{if}\NormalTok{(}\FunctionTok{abs}\NormalTok{(}\FunctionTok{mean}\NormalTok{(params }\SpecialCharTok{{-}}\NormalTok{ old\_params)) }\SpecialCharTok{\textless{}}\NormalTok{ threshold)\{}
      \ControlFlowTok{break}
\NormalTok{    \}}
    
    \CommentTok{\# Record old values}
\NormalTok{    old\_params }\OtherTok{\textless{}{-}}\NormalTok{ params}
    
    \CommentTok{\#if(i \%\% 10 == 1)\{}
    \CommentTok{\#  print(cat("Iter ", i))}
    \CommentTok{\#  print(params)}
    \CommentTok{\#\}}
\NormalTok{  \}}
  
\NormalTok{  dens\_table }\OtherTok{\textless{}{-}} \FunctionTok{matrix}\NormalTok{(}\DecValTok{0}\NormalTok{, }\AttributeTok{nrow=}\FunctionTok{length}\NormalTok{(ys), }\AttributeTok{ncol=}\NormalTok{k)}
  \ControlFlowTok{for}\NormalTok{(j }\ControlFlowTok{in} \DecValTok{1}\SpecialCharTok{:}\NormalTok{k)\{}
\NormalTok{    dens\_table[, j] }\OtherTok{\textless{}{-}} \FunctionTok{dnorm}\NormalTok{(ys, mus[j], sigmas[j])}
\NormalTok{  \}}
  \CommentTok{\#print(dens\_table)}
  
\NormalTok{  col\_names }\OtherTok{\textless{}{-}} \FunctionTok{c}\NormalTok{(}\DecValTok{1}\SpecialCharTok{:}\NormalTok{k)}
\NormalTok{  col\_names }\OtherTok{\textless{}{-}} \FunctionTok{sapply}\NormalTok{(col\_names, }\ControlFlowTok{function}\NormalTok{(j)\{}\FunctionTok{return}\NormalTok{(}\FunctionTok{paste}\NormalTok{(}\StringTok{"Component "}\NormalTok{, }\FunctionTok{toString}\NormalTok{(j)))\})}
  
  \FunctionTok{colnames}\NormalTok{(params) }\OtherTok{\textless{}{-}}\NormalTok{ col\_names}
  
\NormalTok{  clustered }\OtherTok{\textless{}{-}} \FunctionTok{apply}\NormalTok{(dens\_table, }\DecValTok{1}\NormalTok{, which.max)}
\NormalTok{  res }\OtherTok{\textless{}{-}} \FunctionTok{structure}\NormalTok{(}\FunctionTok{list}\NormalTok{(}\AttributeTok{data=}\NormalTok{ys, }\AttributeTok{cluster=}\NormalTok{clustered, }\AttributeTok{params=}\NormalTok{params, }\AttributeTok{dens\_table=}\NormalTok{dens\_table), }\AttributeTok{class=} \StringTok{"EM{-}GMM\_res"}\NormalTok{)}
  \FunctionTok{return}\NormalTok{(res)}
\NormalTok{\}}

\NormalTok{em\_gmm\_model }\OtherTok{\textless{}{-}} \FunctionTok{em}\NormalTok{(data, }\DecValTok{2}\NormalTok{)}

\FunctionTok{print}\NormalTok{(em\_gmm\_model}\SpecialCharTok{$}\NormalTok{params)}

\NormalTok{dt\_1 }\OtherTok{\textless{}{-}}\NormalTok{ data[}\FunctionTok{which}\NormalTok{(em\_gmm\_model}\SpecialCharTok{$}\NormalTok{cluster }\SpecialCharTok{==} \DecValTok{1}\NormalTok{)]}
\NormalTok{dt\_2 }\OtherTok{\textless{}{-}}\NormalTok{ data[}\FunctionTok{which}\NormalTok{(em\_gmm\_model}\SpecialCharTok{$}\NormalTok{cluster }\SpecialCharTok{==} \DecValTok{2}\NormalTok{)]}
\NormalTok{dt }\OtherTok{\textless{}{-}} \FunctionTok{cbind}\NormalTok{(dt\_1, dt\_2)}
\FunctionTok{colnames}\NormalTok{(dt) }\OtherTok{\textless{}{-}} \FunctionTok{c}\NormalTok{(}\StringTok{"Cluster 1"}\NormalTok{, }\StringTok{"Cluster 2"}\NormalTok{)}

\FunctionTok{boxplot}\NormalTok{(dt, }\AttributeTok{n=}\DecValTok{2}\NormalTok{, }\AttributeTok{xlab=}\StringTok{"Clusters"}\NormalTok{, }\AttributeTok{ylab=}\StringTok{"Values"}\NormalTok{, }\AttributeTok{main=}\StringTok{"Box Plot of EM{-}GMM Clustering"}\NormalTok{)}
\end{Highlighting}
\end{Shaded}

\hypertarget{reference}{%
\section{Reference}\label{reference}}

\begin{itemize}
\tightlist
\item
  \href{https://www.aptech.com/resources/tutorials/bayesian-fundamentals/gibbs-sampling-from-a-bivariate-normal-distribution/}{Gibbs
  Sampling from a Bivariate Normal Distribution}
\item
  \href{https://online.stat.psu.edu/stat414/lesson/21/21.1}{21.1 -
  Conditional Distribution of Y Given X}
\item
  \href{https://stats.stackexchange.com/questions/30588/deriving-the-conditional-distributions-of-a-multivariate-normal-distribution}{Cross
  Validated - Deriving the conditional distributions of a multivariate
  normal distribution}
\end{itemize}

\end{document}
